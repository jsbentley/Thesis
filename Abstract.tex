\thispagestyle{plain}
\begin{center}
    \Large
    \textbf{Aspects Associated with the Use of M-dwarf Stars in Exoplanet Searches}\\
    \vspace{0.4cm}
    John Bentley\\
    School of Physics, Faculty of Science, UNSW Australia\\
    \vspace{0.9cm}
    \textbf{Abstract}
\end{center}
Many current exoplanetary surveys focus on stars thought likely to host detectable habitable rocky planets (like the Earth). Selection criteria are required to identify the candidates most likely to host detectable examples of these planets, and to exclude those for which the host star properties may prevent a meaningful detection. This work aims to develop criteria to achieve these goals.\\

Chapter 2 presents a methodology for using large photometric colour and absolute magnitude data sets to identify complete samples of low-mass stars. The result is a set of criteria that uses optical and near-infrared photometry to identify stars most likely to be M-dwarfs, along with an analysis of the false-positive rates for those criteria. This work was published in the Monthly Notices of the Royal Astronomical Society \citep{2019Bentley}.\\

Stars with high levels of spectral variability will tend to also display high levels of apparent radial velocity variation (or ``jitter''), which can make the detection of small exoplanetary signatures difficult.  Chapter 3 describes a new metric (``Intrinsic Stellar Variability'', or ISV) for measuring the spectral variability of a star. In contrast to most current stellar activity metrics (which focus on changes in the line-core emission of select strong spectral lines), the ISV metric method looks at the variation of flux across {\em all} wavelengths in a spectrum. This has the advantage of being applicable to a wide range of stellar classes, as it is not limited to a few strong features. Application of this metric to spectra for 46 M-dwarfs (spanning M0-5.5) identified 84 varying spectral lines, of which 37 were found to vary in at least ten of the stars. These lines are found to be the product of emission in the upper chromosphere. Excluding these wavelengths from radial velocity measurements was found to alter the velocity by \textless1ms$^{-1}$. ISV is useful as an activity metric, but it does not dramatically improve radial velocity measurement precision.\\

Chapter 4 details the adaptation of the photometric selection criteria to also include K-dwarfs and applies this selection to the GALAH survey data. This produces a sample of 6,405 M- and K-dwarfs. Determining the ISV metric for all these stars (as a function of stellar luminosity) allows for an investigation of how the strength and measurement uncertainty of the spectroscopic variation detected by the metric change with absolute magnitude. Stellar variability is expected to influence abundance measurements. Comparison of the abundances of quiescent and active stellar lines (identified via the ISV metric) indicate that the ISV metric may be a potential tool for identifying significantly variable spectral lines that would produce discrepant abundance measurements.\\

