\chapter{Conclusions}
The aim of the work in this thesis was to produce tools that would improve the selection of exoplanet hosts. Two tools were produced - a method of identifying M-dwarfs photometrically, and a new variability metric.\\

The first of these tools enables the production of selection criteria in photometric colours and absolute magnitudes to target a specific class of star. It uses photometry from both real and synthetic data to identify the target class, and selects using colour cuts in multiple colour-colour planes, plus absolute magnitude. This requires a balance optimising for maximal completeness versus minimal contamination. That is, balancing between selecting as many of the target class as possible, while limiting the number of non-target stars. In this case, the class was M-dwarfs, for which the detection of small rocky planets in the habitable zone is expected to be easiest. The location of M-dwarfs in the colour-spaces used meant that the amount of contamination was a small fraction of the number of stars selected, so the criteria were selected to increase the completeness of the M-dwarf population. This resulted the identification of 74,091 potential M-dwarfs brighter than an apparent magnitude of G\,=\,14.5 across the southern sky, avoiding the Galactic plane ($\delta$\,\textless\,+10\,\degree, $|$b$|>$10\,\degree).\\

The second tool was a new metric for identifying activity from observed variability in stellar spectra. The Intrinsic Stellar Variability (ISV) metric measures the variation of each pixel across multiple spectra of a star, and produces a value that is the ratio of the variation in that pixel against the expected variation produced by photometric noise. Applying this to all pixels in the spectrum produces an ISV spectrum that identifies wavelengths that vary at significant levels. The method was applied to HARPS spectra of 46 known M-dwarfs. A threshold for each star was established to identify the wavelengths that vary strongly enough to rule out photometric noise as a cause. Known spectral lines were found at the majority of these wavelengths. While the list of spectral lines identified by ISV varied from star to star, 37 lines were present in at least 10 of the 46 M-dwarfs. The strength of each line was measured and compared to the S-index metrics of the Ca H\&K, H$\alpha$, and Na D lines. There were strong correlations between several of the ISV identified lines and the H$\alpha$ S-index. As these variable lines would have an impact on radial velocity measurements of the star, the radial velocity was measured both with and without the identified lines. Removing these wavelengths produced an increase of up to 20 cms$^{-1}$ in the mean radial velocity uncertainty, but also lowered the average radial velocity scatter about the mean by 10 cms$^{-1}$ in some stars. While the impact of these changes was well below the typical radial velocity precision of 1 ms$^{-1}$, the number of pixels removed was minimal compared to the total wavelength range.\\

The last stage in this research was to apply both tools to select a range of cool dwarfs from the large GALAH database of photometry, spectra, and abundances, and examine how variable lines affect abundance measurements. Because abundances are measured from the equivalent widths of select spectral lines, the standard deviation in the equivalent width of a line within a magnitude bin was compared to the magnitude of the ISV variation from the same line. GALAH includes a comprehensive set of abundance measurements for each star. The photometric selection method was used to identify K- and M- dwarfs in GALAH and an ISV spectrum was produced from all spectra within sixteen 0.2 magnitude wide absolute magnitude bins. Ten lines were found to be common in 5 or more of these bins. \\

The variability of several of these lines was found to change steadily with absolute magnitude. The percent  uncertainty on ISV line strength was found to increase when line strength decreased and vice versa. The biggest impact on measurement uncertainty was found to be the signal-to-noise ratio in the input spectra. The number of observations used to produce an ISV spectrum is significant for ISV measurement uncertainty, but only for bright stars. There is a strong correlation between the strength of an ISV line and the range of equivalent width of the relevant spectral line in a given magnitude bin. Further analysis showed that the strength of an ISV line can be used to estimate the expected scatter when measuring equivalent widths. Lastly, the abundances determined by GALAH for the ten lines found to vary in the ISV of multiple absolute magnitude bins were investigated to determine whether the spectral variability captured by the ISV was a result of stellar variability or an intrinsic range in abundance in the data set. Focusing on just the brightest absolute magnitude bin, we find that the standard deviations in the derived abundances for spectral lines with ISV tend to be higher than for lines without ISV, but the mean abundances are comparable. This indicates that the change in spectral line strength captured by ISV is not causing significant problems for GALAH abundance accuracy, but it may affect precision. However, the set of lines we have available to study is small enough that we cannot make strong claims about whether the ISV in GALAH data is driven more by stellar activity or by abundance scatter in the sample.\\

\section{Future work}
As with any research, new developments lead to more questions that need answering. Some of the potential work that could lead from this research includes:\\

The selection criteria detailed in Chapter\,\ref{ChapPhot} focus on M-dwarfs because they are the best class of stars for detecting rocky exoplanets in the habitable zone. However, they are not the only class of star found to host exoplanets. The selection method should in principle be useful for selecting other classes of stars, but the completeness and contamination may be very different from that found for M-dwarfs. One advantage that M-dwarfs have in most colour-colour planes is that they are fairly well separated from hotter stars. The main source of contamination comes from the blue end of the distribution where the early M-dwarfs overlap with late K-dwarfs. This was advantageous and led to a focus on greater completeness, rather than minimising contamination. A photometric selection of hotter stars would face increased contamination due to overlap with other classes. A focus on completeness over contamination may not be as viable and a more restricted selection may be warranted. The photometric selection method was applied to K-dwarfs and B stars in Chapter\,\ref{chapGALAH}; however, we did not investigate how many of each class were missed, and how many non-K-dwarfs and non-B-stars were included in those samples. Due to the level of astrometric precision available due to Gaia, there are now motion-based catalogues of M-dwarfs available. An important test of the efficiency of the photometric selection method would be to compare the stars selected from the photometric criteria to these catalogues. This would provide more reliable estimates of completeness and contamination. Another aspect to consider is binarity. Binary systems were excluded from the K-dwarf samples used in Chapter\,\ref{chapGALAH} by means of a colour and absolute magnitude limit. Applying this to the M-dwarf selection would improve the quality of the sample. However, as {\em Galaxia} simulates a population of single stars, an estimate of this additional criterion on contamination would not be possible.\\

Wavelengths of significant variation in an ISV spectrum are confirmed to be the product of a varying spectral line via a ``line list'' of well known atomic transitions. The list of spectral lines used to identify the significant variations in the ISV spectra produced by the HARPS data of Chapter\,\ref{chapISV}, and the GALAH data of Chapter\,\ref{chapGALAH}, was not calibrated towards the classes of stars used. Rather, we adapted the line list produced by the GALAH survey, which is attuned towards Solar-type stars. While it was possible to identify multiple varying spectral lines in the HARPS and GALAH datasets, there were many ISV lines that were neglected in this analysis as those lines were not in the line list. Using a list of all known transitions, such as the information provided by the National Institute of Standards and Technology (NIST), could significantly increase the number of identified ISV lines. However, many transitions listed in NIST's database would be impossible in an M-dwarf atmosphere. Instead of simply replacing the GALAH line list with the full NIST database, a further refinement would be required to determine which of the nearby lines is most likely to have caused the variation seen in the ISV spectrum.\\

Removal of the ISV lines from the HARPS M-dwarf spectra in Chapter\,\ref{chapISV} had limited influence on radial velocity precision. It is expected that this is due to the limited number of pixels removed, and that if additional variable spectral lines were identified, their removal could potentially improve the cross-correlation and therefore the radial velocity measurements. As mentioned in the previous paragraph, the line list used to identify spectral lines was limited in its use for M-dwarfs. A more comprehensive line list would potentially identify more ISV lines, which would expand the range of wavelengths removed from radial velocity measurements.\\

The ISV spectra produced in Chapter\,\ref{chapISV} were based on M-dwarf spectra from HARPS. HARPS has been observing these stars to look for the periodic radial velocity signature that indicates the presence of a planet orbiting the star. The spectra need to be velocity corrected to the same rest frame to produce a meaningful ISV spectrum. If a star `wobbles' due to the influence of an exoplanet, that may interfere with the ISV process and produce variation that is not due to changes to the stellar atmosphere. If it does, how large does the semi-amplitude of the radial velocity variation have to be before it becomes an issue? To investigate this, stars with known planets should have their planet's radial velocity signatures removed before the ISV spectrum is produced. Changes in the resulting ISV spectrum could inform as to how much an exoplanet influences the ISV process and whether it needs to be a consideration.\\

The non-flat structure seen in the baseline noise of the HARPS ISV spectrum is expected to be a product of the decision to use the square root of the flux as an estimate of the photometric uncertainty. A flatter baseline noise would lead to a lower ISV line selection threshold, which should then include more of the weaker, but still statistically significant, ISV lines. To investigate this, an ISV spectrum needs to be produced with a more sophisticated algorithm for estimating photometric uncertainties, including the impacts of calibration uncertainties due to flat-fielding, blaze correction, continuum fitting, sky subtraction, etc. Improvements to the algorithm should also improve the baseline noise in ISV spectra like those shown in  Figure\,\ref{figGALAH_B_1} and \ref{figGALAH_B_3}, where the profile of strong lines like H$\alpha$ and H$\beta$ can also produce non-flat structure in the baseline noise.\\

The magnitude of variation found in ISV spectra was compared to the S-indices of the Ca\,\textsc{ii} H \& K, H\,\textsc{$\alpha$}, and Na\,\textsc{i} D lines and some correlation (particularly with H\,\textsc{$\alpha$}) was found. However the S-indices have their limitations, and there are alternative methods for measuring the variability of stars. One such alternative metric by \citealt{2012Bell} uses the scatter in equivalent width measures for H$\alpha$ (normalised by the mean value of those equivalent width measures), to determine the relative strength of variability as an activity metric. Specific activity metrics tend to be sensitive to activity in specific regions of the stellar atmosphere. Comparing the ISV variation to this method (and others) would better inform us as to what types of activity the ISV method is measuring.\\

Due to the limited number of observations per star, the ISV spectra produced in Chapter\,\ref{chapGALAH} used observations of multiple stars falling within a narrow absolute magnitude range rather than multiple observations of individual stars, which reduced the star-to-star variation in the spectra. Despite this, there will still be some variation due to varying metallicities. To refine the selection further, a colour range for each bin may be advantageous. This would reduce the number of observations per bin, which would produce a lower quality ISV spectrum. However as the spectra in each bin would now be more similar, this may offset the reduction in quality from the reduced number of observations. Particularly, for the brighter absolute magnitude bins, the number of observations is sufficiently large that a reduction may be entirely beneficial. An additional consideration would be stellar mass. Limiting the spectra used to produce an ISV spectrum to stars of a limited mass range, and observing how the ISV spectra changes with changing mass could provide interesting information about variability and mass.\\

Comparing the abundances in K-dwarfs derived from Ti and V lines with and without ISV, the ISV line abundances were found to not be systematically offset from the non-ISV line abundances. This indicates that the variability measured in ISV is not significantly affecting the derived GALAH abundances. Across the brightest absolute magnitude bin, the standard deviation in the abundances of those same Ti and V lines is not significantly different between the lines that do and do not show ISV. We interpret this to mean that abundance variation within the data set is responsible for the variable line strengths found by our ISV analysis. Collecting enough observations ($\geq$20) of individual stars with a detailed set of abundance measurements and producing ISV spectra for them would provide an opportunity for this situation to be clarified, since the true abundances of the star would be the same in each observation.\\